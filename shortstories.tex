\documentclass[12pt, justified, a4paper, symmetric]{tufte-book}

\usepackage[T1]{fontenc}
\usepackage{librecaslon} % use the caslon font. alternatives are palatino, stix
\usepackage{microtype}

\setlength{\parindent}{18pt} % disable paragraph indentation
\setlength{\parskip}{6pt} % space after par increased from 0 to 6 pt
\renewcommand{\baselinestretch}{1.1} % line spacing increased by 10%

% chapters and sections
\usepackage[raggedright]{titlesec}
\usepackage{blindtext}
\usepackage{color}

\titleformat{\chapter}[hang] % format chapter
	{\huge\mdseries\raggedright}
	{}
	{0pt}{\huge\mdseries}

\titleformat{\section}[hang] % format section
	{\large\bfseries\raggedright}
	{}
	{0pt}{\large\bfseries}

\begin{document}
\section{God ziet U, hier vloekt men niet}
De omineuze spreuk sierde, onder begeleiding van een alziend oog, de woonkamer. Een dubbele rij sanseveria's op de vensterbank moest er leven en jeugdigheid brengen, maar hun falen was bijna spectaculair. Het hele huis ademde een oerbanaliteit die het er voor een stadsjong zoals mezelf vanzelfsprekend interessant maakte. Vele zondagen had ik hier gesleten, delvend tussen de zilveren zakhorloges, medailles en en trofee\"en. Je hoefde er de eiken dressoirs maar open te trekken, en je was er een hele middag zoet. Vaak zelfs letterlijk.

Een onuitputtelijke bron aan clich\'es kon aangesneden worden om de plek te beschrijven: "Je voelde je er weer kind", of "Het leek alsof de tijd er had stilgestaan." Maar de tijd stond er niet stil. Bijna klagend klikte de staande klok verder. Nergens werd men zo geconfronteerd met de vergankelijkheid als hier. De aanwezigheid van de dood was er verzengend - van de verzameling bidprentjes op het schouwblad tot de gekruisigde godszoon aan de muur. En de bewoners zelf, natuurlijk. Spoedig zou ook hun bidprentje in iemands collectie belanden.

Ze zaten elk in een hoek van de kamer, hun lichamen aan het afronden, maar geestelijk nog steeds aanwezig. Achter de vrouw schuilde een replica van De Nachtwacht in kruissteek. Nu kon ze zoiets niet meer. Haar zicht was versleten. Bij elk gesprek tuurde ze star voor zich uit, ongeacht de bron van het geluid. Het zoeken was al lang verleden tijd. Hij zat in een leunstoel tegenover haar. Vroeger had de leren eenzit goed opgevuld geweest, maar vandaag zakte zijn bewoner diep weg. De eens zo donderende, zwaargezette man ging nu op in het decor.

Moeizaam leunde hij voorover en richtte zich tot "zijn gastjes". "D'r is nen tijd van komen, en d'r is nen tijd van gaan", sprak hij, "en den tijd van gaan is Godverdomme aangebroken. Begrijpt ge mij?" Hij meende het. De gastjes, al lang volgroeide gasten geworden, knikten zacht. Begrepen ze hem echt? Bij haar was een reactie zoek. Ook zij wist het. Ze zou ondergaan. De angst en het taboe waren weg.

God zag hen, ook al hadden ze gevloekt.

\newpage
\section{Een oude vriend}
Mijn eerste bezoek aan Bert speelde zich af ergens in april, een kleine tien jaar geleden. Ik was een jaar of negen en onverzadigbaar nieuwsgierig naar dieren en de natuur. We hadden een zaterdagvoormiddag afgesproken waarop ik bij hem thuis op visite mocht komen. Op m'n eentje was ik naar zijn huis gefietst. Ik had me een weg gebaand door hectares tarwe en ma\"is, tussen twee Oost-Vlaamse dorpen in, om uiteindelijk aan te komen bij een oudere, maar goed onderhouden hoeve.

Zoals afgesproken wandelde ik langs het open poortje naar de achterzijde van het huis. Via een kiezelig pad vond ik vervolgens de weg door de siertuin, dan richting de achtertuin en verder door naar de boomgaard en de moestuin. Daar trof ik hem aan op beide knie\"en, met zijn handen ploeterend in de vochtige voorjaarsgrond. Hij was jonge kolen aan het verplanten. De tenen van z'n laarzen zaten diep in de zachte aarde verzonken. Toen hij me opmerkte, beschermde hij met de ene hand zijn ogen tegen de rijzende zon en maande hij me met de andere aan om nader te komen.

"Florian, hoe is 't er mee? Ik kan u anders snel een rondleiding geven door de tuin." Er was geen plaats voor protest. Hij stopte de laatste savooi in, stond op en nam me mee op avontuur. Achterin de tuin troffen we twee bijenkorven aan. "Nieuwe van vorig jaar, de oude werden te klein." Zonder aarzeling hief hij het houten deksel op om me de inhoud van de korf te laten zien. De deken van bijen dat daarbij neerstreek op zijn hand deerden hem niet - de beestjes leken hem te kennen. "Als ge wilt, kunt ge straks ook wat honing meenemen voor thuis", verkondigde hij fier. Ook daar verwachtte ik geen plaats voor protest.

Toen we verder door de boomgaard wandelden, toonde hij zijn pas ge\"ente kerselaars. Vorige winter had hij jong griffelhout verzameld en hier op oude onderstammen aangebracht. Het kronkelende onderhout zelf was al vele generaties deel van het nalatenschap. De resultaten van zijn werk zouden pas binnen tien \'a vijftien jaar geproefd kunnen worden, maar hij had nog tijd. Iets verder kwamen we aan bij een groot uitgebouwd vogelkot. Binnenin werd ons gesprek overstemd door het vrolijke gekwetter van de bewoners. Het rook er naar hooi, houtzaagsel en zaden, maar was er voor de rest onberispelijk schoon. Door de zuinige zonnestralen en het wentelende stof was de zichtbaarheid er beperkt, maar Bert wist zonder moeite uit de fladderende chaos zijn favoriete zangvogeltje te grijpen. Vol trots hield hij het bedeesde dier in beide handpalmen naar mij gericht. Toen ik naderde om te aaien, spreidde het vogeltje ras de vleugels. Dit vond hij bijzonder geestig, en al lachend sloten we de deur achter ons. In een ren naast het kot woonden ook enkele fazantenhennen en \'e\'en enkele haan. Met hen kweekte hij de heerlijkste nieuwjaarsdiners. Ook door deze vogels werd hij gauw herkend; ze kwamen zijn hand graag tegemoet toen hij een handvol ma\"iskorrels neerstrooide. Zonder aarzelen kropen ze tegen zijn gehurkte laarzen aan. Ook zij vluchtten toen ik hun territorium trad. We sloten het kippengaas achter ons en wandelden verder door naar het pronkstuk van de rondleiding: de siertuin. De voorgrond was bezaaid met viooltjes, meiklokjes en andere klassiekers. Elke tweede pas hield hij halt om een rups van het bloembed te plukken of onkruid te ontwortelen. Telkens ik een plant uitpikte, wist hij ze bij naam te noemen, af en toe zelfs met de Latijnse.

Aan het einde van de tuin toonde hij me een schommelstoel voor \'e\'en persoon. Hij vertelde me hoe heerlijk vertoeven het er 's zomers was, onder de ondergaande zon met een kan koele kruidenthee. Hoe de heldere hemel er plaats maakte voor het sterrenbed, en de maan en de krekels hem gezelschap hielden. Ik zou er zelf nog vele zomeravonden zo zitten, turend naar het hemelgewelf, de sterren tellend.

Aan het einde van de rondleiding kwamen we bij de veranda, waar we in de schaduw plaatsnamen. Vanuit de keuken werden we vergezeld door Willem Vermandere en Jan De Wilde. Niet veel later kwam zijn vrolijke grootmoeder aan met een plateau honingkoeken en melk. Uit zijn boekentasje diepte Bert een pennenzak en een stapeltje papier op. We hadden immers nog een hele middag werk aan onze spreekbeurt voor biologie.

\newpage
\section{Uitgeput}
De allereerste keer dat ik meereed, lag het kampterrein in Gedinne, een uithoek van de Ardennen ergens ten zuiden van Rochefort. Onze traditie dicteert immers dat je vanaf je 14 jaar met de fiets gaat kamperen. 's Ochtend vertrekken, nog voor de zon opkomt, en 's avonds om tien uur bekaf arriveren. En daarna mag je nog gaten beginnen boren voor je tent. De afstand Buggenhout/Gedinne bedraagt per fiets een kleine 170 kilometer.

We hadden het leeuwendeel van de tocht achter ons. Ik bevond me in een eindeloze parade van bergen en dalen. Het zweet parelde. Mijn wenkbrauwen voelden vochtig. Strak over het stuur gebogen had ik uitzicht op mijn vrienden verderop. Zoals altijd zat ik achter, maar zolang de afstand niet te groot werd, zouden we niet stoppen. Elke ingehaalde centimeter was een persoonlijke overwinning. De afstand constant houden was net geen persoonlijke nederlaag, maar ook daar nam ik al genoegen mee. Door de veel te losse rugzakriemen waren er brandwonden ontstaan op mijn beide schouders. De logge zak maakte een cyclische beweging van links... Naar rechts... En weer naar links... Naar rechts...

En... Stop. Even pauze. Ik stapte af en wierp met een luide zucht de last af. Over mijn doorweekte rug blies nu een aangenaam briesje. Met een losse beweging goot ik drinkwater over mijn gloeiende schouders. Ik zag hoe in de verte nu ook de kopgroep moeite kreeg met het temmen van de Ardense heuvels. Enkelen onder hen hadden al afgestapt en gingen met de fiets in de hand te voet verder. Er vormde zich een blauwe kolonne jeugdbewegingsuniformpjes. Met dit beeld keerde de moed weer. Ik spande de schouderriemen haastig aan en kroop weer op het zadel. Door de strakke schouders vlotte het trappen v\'e\'el beter. Vol vertrouwen schakelde ik naar een hogere versnelling. Ik haalde zelfs een ouwe knar met een pet in! Trap voor trap keerde de afstand tussen het peloton en mezelf weer terug naar een normaal peil.

Intussen bereikte de kolonne echter de top van de heuvel. Stilaan stapten de uniformpjes weer op, en \'e\'en voor \'e\'en verdwenen ze over de horizon. In het steile deel van de beklimming begon ik het weer moeilijk te krijgen. De schouders waren zwaar, ik voelde me licht. In mijn gedachten riep ik: "H\'e, wacht! Ik ben er ook nog!" Stilte. Waarom keek niemand om? Waren ze me dan vergeten? Een verkropte huilbui baande zich een weg door mijn keel. Zweetdruppels werden zachtjes vervangen door tranen. Ik voelde mezelf tot stilstand komen. Met beide voeten op de grond leunde ik over het stuur. Ik gaf op.

Tot ik plotsklaps vooruitschoot. Geschrokken keek ik achterom en zag het gezicht van de man die ik zonet voorbij had gestoken. Zijn magere, gerimpelde wangen en gedateerde kledij plaatsten hem ergens rond de zeventig, maar zijn ogen waren vol vuur. Met zijn rechterhand hield hij stevig zijn stuur vast, met de linker stuurde hij mij. In paniek begon ik te trappen, maar mijn pedalen bewogen haast uit zichzelf. Met een onverwachte kracht en snelheid trok de man ons beiden de heuvel op.

Aan de top duwde hij me nog met een laatste adem sterk vooruit zodat ik op mijn eentje naar beneden rolde. "Vas-y, petit gamin!", riep hij me na.

Ik heb nooit een woord gezegd, genoegen geuit of zijn heldendaad erkend. Ik keek niet om en toonde geen teken van dank. Mijn hoofd was licht en leeg. Ik verkeerde in een etherische, bovenlichamelijke sfeer. Zachtjes rolde ik door, zonder te trappen. Ik hoorde enkel mijn eigen ademhaling. Aan de voet van de heuvel werd ik opgewacht door mijn vrienden. Met enkele schouderklopjes en zonder veel woorden vertrokken we weer, en aan een iets trager tempo maakten we de tocht samen af.

\newpage
\section{Jeugdig}
Deze herinnering speelt zich af op een doffe zomeravond, een tiental jaar geleden. Ik fietste terug naar huis van een brasserie in Beauraing, het naburig dorp waar ik buiten de boeken bijkluste als ober. Communies, trouwfeesten, vaak nog tot in de vroege uurtjes. Toen lukte dat nog allemaal, pas op pensioen en vol energie. Daarvoor ben ik nu te oud, maar bij het ophalen van deze herinnering keert dat oude gevoel van jeugd weer terug.

Die dag had het tropisch warm geweest. Onder de lage zon wierpen de bossen en de heuvels lange, uitgestrekte schaduwen neer. Het hemellicht doofde zachtjes uit en stilaan nam de maan over. Zelfs zonder zon was het buiten nog behoorlijk warm. Het gladde asfalt glinsterde nog na van de middaghitte, en de oude, rubberen handgrepen van mijn stuur voelden zacht en plakkerig aan. Het op-en-neer-deinende landschap deed mijn oude rijwiel bij elke trap zachtjes piepen.

In \'e\'en van de steile afdalingen stak een groepje jongeren in jeugdbeweginstenue me langs links en rechts voor. Het was een vrolijke bende. Ze waren duidelijk uitgelaten om na de steile beklimming weer af te dalen. Om het hardst zoefden ze, allen zwaarbeladen met kampeergerei, naar beneden. Zelfs zonder hun Vlaams gejuich was het duidelijk dat ze niet van hier waren. Iedereen afkomstig van de streek weet immers dat je de afdaling rustig aan neemt en zo energie spaart voor de volgende heuvel. Rustig van de berg rollen en met constante snelheid weer omhoog, zo moet het. Met een grijns verheugde ik me er al op om hen binnen enkele kilometers weer voorbij te steken. En ja, een tiental minuten later zag ik van op de vlakte een mooi rijtje blauwe uniformpjes. Pas na pas slenterden ze met hun fiets in de hand de berg op. Hoewel ze al goed gevorderd waren, schatte ik dat ik hen nog net voor de top zou kunnen inhalen.

Plots dook achter mij nog een nakomertje op. Vol overtuiging stak ook hij me in een sprint voorbij. Hij trapte hard door, alsof hij een aanloop nam om de berg daarna in \'e\'en keer op te rijden. Dat zou hem sowieso niet lukken. Op het steilste deel begon hij noodgedwongen te vertragen. In een poging tot weerstand ging hij rechtstaan op zijn pedalen. Intussen waren zijn vriendjes bijna halfweg de heuvel. Ikzelf deed het nog steeds goed - mijn oude fiets beschikte niet over versnellingen, maar dat was ook niet nodig. Aan dit tempo zou ik gegarandeerd de top bereiken voor de jeugdige bende. Glimlachend zou ik hen dan \'e\'en voor \'e\'en voorbijsteken.

Intussen naderde ik het nakomertje. Hij was inmiddels bijna tot stilstand gekomen. Hopeloos trapte hij in lage versnelling verder. Op enkele meters afstand hoorde ik zijn gedempte gesnik. Tussen elke ademhaling door riep hij zijn vrienden zachtjes om hulp. Mijn geniepige zin op wraak werd vervangen door zacht medelijden en begrip. Bewogen besliste ik het plan om de rest in te halen te schrappen en dit arme kind verder te helpen. Met \'e\'en vloeiende beweging pakte ik hem bij de arm en trok hem met me mee. Geschrokken keek hij achterom, zijn ogen vol opluchting en dank.

Met veel moeite bereikten we samen de top van de heuvel. Een laatste duw stuurde hem naar beneden. "Vooruit, kleine man!", riep ik hem nog na, en daarmee rolde hij stil de heuvel af. Zuchtend rustte ik even uit op het stuur. Met mijn pet wreef ik de zweetdruppels van mijn voorhoofd. Ik voelde me bekaf, maar springlevend. In de verte werd het nakomertje opgewacht door zijn vrienden. Na een korte verbroedering trokken ze weer allen samen verder. Ikzelf moest de andere richting uit. Voor het vertrek keek, ik nog even achterom of ik alleen was. Vervolgens scheerde ik met een luide vreugdekreet zo hard ik kon de berg af.

\newpage
\section{Heldin}
Ik keek naar mezelf in de spiegel. Met mijn vingers betastte ik de vele rimpels en vlekken. Door de vermoeide, donkere kringen onder mijn ogen zag ik er jaren ouder uit dan ik in werkelijkheid was. Hoe had het toch zo ver kunnen komen? Geen geld, geen familie, geen toekomst. Geen hero\"ine. Reddeloos. Ik dimde het licht tot de kamer halfdonker was. In de schemering leken mijn uitgedunde haren en bleke gezicht een beetje gezonder. Zo val ik nog wel mee, dacht ik.

Door het deurgat zag ik hoe Seppe op de vloer tv keek. Ik durfde hem niet naar de kleuterschool te sturen uit angst dat iemand aan de alarmbel zou trekken. Als ik hem zou verliezen, was het gedaan. Hij was alles, al kon ik hem nog met moeite onder controle houden. Soms begon hij plots, zonder duidelijke reden, te roepen en te krijsen. Dan kon hij niet meer getroost worden, schuwde hij elke aanraking. Een gloeiende schaamte rees me naar het hoofd en met moeite onderdrukte ik de zoveelste huilbui.

Diep ademhalen, dacht ik, niet nu - binnen enkele ogenblikken is hij hier. Deze keer had ik hem zelf gebeld. Het lukte niet langer om zonder te zitten. Hij was bekend onder de meisjes, sommigen waren volledig afhankelijk van hem. Hij kon aan alles geraken, maar zelf gebruikte hij niet. Eerst kocht ik gewoon en hadden we nauwelijks contact, tot ik niet meer kon betalen en we... nu ja...

Klop-klop. Ik haalde opnieuw diep adem en stond op om de deur voor hem te openen.
"Goeienavond!", groette hij verheugd, "Mag ik binnenkomen?" Verslagen probeerde ik een glimlach te forceren. Nee, idioot, ik heb je een uur geleden gebeld maar nu mag je niet binnen. Hij stapte door de deur, veegde zijn voeten en keek even rond. "Alles ok hier? En met jou, Sep?" Geen reactie. Dat deed hij altijd - vals ge\"interesseerd vriendelijk proberen te lijken. Mijn zoontje lastigvallen. Toestemming vragen om binnen te komen. Alsof dit allemaal mijn eigen keuze was.

"Ik heb iets lekkers meegebracht", grinnikte hij. Uit zijn zakken haalde hij enkele oude lekstokken tevoorschijn en legde ze op tafel. "En voor de mama heb ik ook iets lekkers mee", fluisterde hij in mijn oor. Uit zijn andere zak diepte hij een klein balletje bruine hero\"ine op. Mijn mond werd droog. Daarvoor wilde, en zou, ik alles geven. "Maar da's voor later", en hij stopte het zakje weer weg. Met zijn ogen maande hij me naar de slaapkamer.

Voor ik hem volgde, wilde ik de tv nog iets luider zetten. Nog voor ik aan de knoppen kon, kermde Seppe ge\"irriteerd dat ik van voor het scherm moest blijven. Geschrokken keek ik hem in de ogen, zoekend naar liefde. Niets.

Op dat moment wilde ik voor het eerst sterven. Alles laten eindigen, zonder pijn of geluid. In de donkere, koude nacht met een lange, lange aanloop van het dak springen. Het klonk heerlijk. Maar de drang naar een shot was sterker. Ik draaide me om naar de slaapkamer.

Eens we alleen waren, wond hij er geen doekjes meer om. "Kleren uit en op het bed." Vol schaamte kleedde ik me uit en ging ik liggen. Hij draaide het licht in de slaapkamer weer omhoog en inspecteerde me. "God, wat is je lichaam knap. Maar vergeet niet af en toe te eten. En die kleine, geef je hem wel te eten?" "Hou hem erbuiten en schiet op", sneerde ik piepend. Alsjeblief... Hij trok zijn ogen op. "Zoals jij het wil", antwoordde hij, en hij zakte over me heen. Hij hield zich niet in. Met gesloten ogen luisterde ik naar het gedempte geluid van de tv. Ik voelde zijn warme adem naderen, maar draaide mijn hoofd weg.

Toen ik mijn ogen opende, zag ik mezelf liggen in de spiegel. Dit doet me allemaal niks, dacht ik, dit is ok. Straks is alles voorbij.

Het dak lonkte.
\end{document}
